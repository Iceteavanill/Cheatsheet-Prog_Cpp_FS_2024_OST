\section{Makefiles}

Makefiles sollten generell das Umsetzten des Codes in Maschinencode vereinfachen.
Es ist eine \say{Skriptingsprache} welche grob nach dem Muster \say{Erzeugnis/target} : \say{Abhängigkeiten/Dependency} folgt. 
Dann folgt indentiert der Befehl, um dieses Erzeugnis zu erzeugen.\\
Wenn ein Erzeugnis keine Datei zurückgibt, muss dieser als \say{.PHONY} markiert werden. 
Das verhindert, dass eine Datei mit demselben Namen das Ausführen verhindert.\\
Am besten sieht man das an einem Beispiel:

\lstinputlisting[language = Make, frame=single]{code/make1}

Der Befehl \say{make all} baut nun das Projekt.\\
Sind die o Dateien noch aktuell werden diese nicht erneut kompiliert. 
Das Projektverzeichnis kann einfach durch \say{make clean} aufgeräumt werden.

\subsection{Platzhalter}

In einem Makefile können auch Platzhalter verwendet werden:

\begin{itemize}[itemsep=1pt, parsep=0pt]
    \item \textbf{\$ @} Dateiname des Targets
    \item \textbf{\$ $<$} Dateiname der ersten Dependency des aktuellen Targets
    \item \textbf{\$ $\wedge$} Dateiname aller Dependencies des aktuellen Targets durch Leerzeichen getrennt
\end{itemize}

Mit diesen Mitteln kann ein Professionelleres Makefile erstellt werden, welches relativ universell eingesetzt werden kann:

\lstinputlisting[language = Make, frame=single]{code/make2}

\nextcol

\section{Styleguide}

\begin{itemize}[itemsep=1pt, parsep=0pt]
    \item \textbf{Variablen, Konstanten}
    \begin{itemize}[itemsep=1pt, parsep=0pt]
        \item mit Kleinbuchstaben beginnen
        \item erster Buchstaben von zusammengesetzten Wörtern ist gross (mixed case)
        \item keine Underscores
    \end{itemize}
    Beispiele : counter, maxSpeed
    \item \textbf{Funktionen}
    \begin{itemize}[itemsep=1pt, parsep=0pt]
        \item mit Kleinbuchstaben beginnen
        \item erster Buchstaben von zusammengesetzten Wörtern ist gross (mixed case)
        \item Namen beschreiben Tätigkeiten
        \item keine Underscores
    \end{itemize}
    Beispiele : getCount(), init(), setMaxSpeed()
    \item \textbf{Klassen, Strukturen, Enums}
    \begin{itemize}[itemsep=1pt, parsep=0pt]
        \item mit Grossbuchstaben beginnen
        \item erster Buchstaben von zusammengesetzten Wörtern ist gross (mixed case)
        \item keine Underscores
        \item Namen beschreiben Dinge
    \end{itemize}
    Beispiele : MotorController, Queue, Color
\end{itemize}
