\section{Basics}

\subsection{Trivia}

Entstanden durch Stroustrup 1986.

\subsection{Grundsätzlich}

C++ ist sehr ähnlich zu C. Es gibt aber einige wichtige Änderungen zu C:

\begin{itemize}[itemsep=1pt, parsep=0pt]
    \item zu benutzender Compiler heisst jetzt \say{clang++}
    \item Nativen Bool Typ : \textbf{bool}
    \item richtiges Konstanten Schlüsselwort : \textbf{constexpr}
    \item Standartbibliotheken haben kein .h mehr beim Aufrufen
    \item C Standartbibliotheken Können weiter verwendet werden, haben aber ein C im Header(stdio.h $\rightarrow$ cstdio)
    \item Char-Literale (z. B. 'A') haben in C++ den Datentyp char (in C war es int)(Brauchen ein const)
    \item Es gibt benannte Namensräume. Wichtigster Namensraum für die STL: std
    \item For Each loop support (siehe Kapitel \ref{forEach})
    \item ...
\end{itemize}

\subsection{Sourcecode Hello world CPP}

\lstinputlisting{code/hello_World.cpp}

\subsection{For each Loop}\label{forEach}

For each Loops laufen automatisch ein Array ab, ohne das zuerst die Grösse ermittelt werden muss. 
Die C for loop syntax wurde etwas erweitert.
Syntax:

\lstinputlisting{code/forEach.cpp}

Siehe: Als Array Name wird ein Pointer auf das Array verlangt, nicht auf das erste element!  
Ein foreach funktioniert \textbf{nicht} mit einem \textbf{Pointerpointer}. 

\subsection{Streams}

Ein Stream repräsentiert einen generischen sequentiellen Datenstrom. Z.b: ein Eingabefeld, Dateien oder Netzwerktraffic.\\
Die wichtigsten Operatoren sind:

\begin{itemize}[itemsep=1pt, parsep=0pt]
    \item \say{$<<$} Inserter $\rightarrow$  Daten einfügen
    \item \say{$>>$} Extractor $\rightarrow$ Daten herausholen
\end{itemize}

Für Standardklassen sind diese Operatoren bereits definiert.

\subsubsection{Standardstreams}

\begin{itemize}[itemsep=1pt, parsep=0pt]
    \item \textbf{cin} : Standard Eingabe
    \item \textbf{cout} : Standard Ausgabe
    \item \textbf{cerr} : Standard Fehlerausgabe, möglicherweise mit clog gekoppelt 
    \item \textbf{clog} : Standard Log Ausgabe 
\end{itemize}

\subsubsection{Streamformatierung}

Formatierungen der Streams können mit folgenden Schlüsselwörtern erreicht werden:


\begin{center}
    \begin{tabular}{cc}
        \rowcolor[RGB]{239,239,239} 
        \textbf{Flag} & \textbf{Wirkung}                                        \\ \hline
        boolalpha     & bool-Werte werden textuell ausgegeben                   \\
        dec           & Ausgabe erfolgt dezimal                                 \\
        fixed         & Gleitkommazahlen im Fixpunktformat                      \\
        hex           & Ausgabe erfolgt hexadezimal                             \\
        internal      & Ausgabe innerhalb Feld                                  \\
        left          & linksbündig                                             \\
        oct           & Ausgabe erfolgt oktal                                   \\
        right         & rechtsbündig                                            \\
        scientific    & Gleitkommazahl wissenschaftlich                         \\
        showbase      & Zahlenbasis wird gezeigt                                \\
        showpoint     & Dezimalpunkt wird immer ausgegeben                      \\
        showpos       & Vorzeichen bei positiven Zahlen anzeigen                \\
        skipws        & Führende Whitespaces nicht anzeigen                     \\
        unitbuf       & Leert Buffer des Outputstreams nach Schreiben           \\
        uppercase     & Alle Kleinbuchstaben in Grossbuchstaben wandel
    \end{tabular}
\end{center}

Folgende Funktionen setzten Header \textbf{$<$iomanip$>$} voraus:\\

\textbf{setw(n)}\\

Wenn diese Funktion mit einer ganzen Zahl dem Cout übergeben wird, wird Cout \textbf{mindestens so viele Zeichen} übergeben. 
Falls Zeichen Fehlen werden diese mit dem Füllzeichen (standardmässig das Leerzeichen) aufgefüllt. 
Diese Funktion setzt die weite nur \textbf{einmalig}.\\

\textbf{setprecision(n)}\\

Diese Funktion setzt die Anzahl Dezimalstellen (vor und nach dem Komma) welche ausgegeben werden. 
In kombination mit \textbf{fixed} werden nur die Anzahl \textbf{Nachkommastellen} gesetzt. 
Die gegebenen Werte bleiben über mehrere einfüge Operationen erhalten (muss man nur einmal setzen). 




