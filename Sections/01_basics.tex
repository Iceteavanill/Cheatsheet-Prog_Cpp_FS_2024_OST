\section{Basics}

\subsection{Trivia}

Entstanden durch Stroustrup 1986.\\

\subsection{Grundsätzlich}

C++ ist sehr ähnlich zu C. Es gibt aber einige Wichtige Änderungen zu C:

\begin{itemize}[itemsep=1pt, parsep=0pt]
    \item zu benutzender Compiler heisst jetzt \say{clang++}
    \item Nativen Bool Typ : \textbf{Bool}
    \item richtiges Konstanten Schlüsselwort : \textbf{constexpr}
    \item Standartbibliotheken haben kein .h mehr beim Aufrufen
    \item Char-Literale (z. B. 'A') haben in C++ den Datentyp char (in C war es int)
    \item Es gibt benannte Namensräume. Wichtigster Namensraum für die STL: std
    \item ...
\end{itemize}

\subsection{Sourcecode Hello world CPP}

\lstinputlisting[language = c++]{code/hello_World.cpp}

\subsection{Streams}

Ein Stream repräsentiert einen generischen sequentiellen Datenstrom. Z.b: ein Enigabefeld, Dateien oder Netzwerktraffic.\\
Die wichtigsten Operatoren sind:

\begin{itemize}[itemsep=1pt, parsep=0pt]
    \item \textbf{$<<$} Inserter $\rightarrow$  Daten einfügen
    \item \textbf{$>>$} Extractor $\rightarrow$ Daten herausholen
\end{itemize}

Für definierte Klassen sind diese Operatoren bereits definiert.

\subsubsection{Standardstreams}

\begin{itemize}[itemsep=1pt, parsep=0pt]
    \item \textbf{cin} Standart Eingabe
    \item \textbf{cout} Standart Ausgabe
    \item \textbf{cerr} Standart Fehlerausgabe
    \item \textbf{clog} Mit cerr gekoppelt aber etwas anders
\end{itemize}

\subsubsection{Streamformatierung}

Formatierungen der Streams kann mit folgenden Schüsselwörtern erreicht werden:


\begin{center}
    \begin{tabular}{cc}
        \rowcolor[HTML]{EFEFEF} 
        \textbf{Flag} & \textbf{Wirkung}                                        \\ \hline
        boolalpha     & bool-Werte werden textuell ausgegeben                   \\
        dec           & Ausgabe erfolgt dezimal                                 \\
        fixed         & Gleitkommazahlen im Fixpunktformat                      \\
        hex           & Ausgabe erfolgt hexadezimal                             \\
        internal      & Ausgabe innerhalb Feld                                  \\
        left          & linksbündig                                             \\
        oct           & Ausgabe erfolgt oktal                                   \\
        right         & rechtsbündig                                            \\
        scientific    & Gleitkommazahl wissenschaftlich                         \\
        showbase      & Zahlenbasis wird gezeigt                                \\
        showpoint     & Dezimalpunkt wird immer ausgegeben                      \\
        showpos       & Vorzeichen bei positiven Zahlen anzeigen                \\
        skipws        & Führende Whitespaces nicht anzeigen                     \\
        unitbuf       & Leert Buffer des Outputstreams nach Schreiben           \\
        uppercase     & Alle Kleinbuchstaben in Grossbuchstaben wandel         
    \end{tabular}
\end{center}



